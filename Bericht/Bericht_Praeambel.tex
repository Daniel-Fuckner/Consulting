\documentclass[12pt,a4paper]{scrartcl} 

%Deutsch:
	\usepackage[ngerman]{babel} %Deutsches Datumformat, Umlaute m\"oglich,...
	\usepackage[latin9]{inputenc} %Umlaute direkt eingebbar
	
%Quellenverzeichnis:
	\usepackage[autocite=footnote,uniquename=full,uniquelist=true,backend=bibtex8]{biblatex}
	\usepackage{csquotes} %Hilfspaket für Biblatex
	\bibliography{bib_database.bib} %Datei mit bibliographischen Daten

%Mathematik:
	\usepackage{dsfont} %Symbole
	\usepackage{amsmath} %Umgebung
	\usepackage{amssymb} %Symbole
	\usepackage{bbm} %doppelstreifen bei buchstaben (zb symbol für ganze zahlen \mathbbm{Z})
	
%Grafiken:
	\usepackage{graphics}
	\usepackage{graphicx}
	\usepackage{picinpar} %bilder so einfügen, dass text um bilder weiterläuft
	\usepackage{float} %\begin{figure}[H] => Grafik wird HIER eingefügt!

%Quellcode:
	%\usepackage[numbered,framed]{mcode} %Quellcode darstellen
	
%Englisch:
	%\usepackage[ngerman,english]{babel} %automatisch erzeugte Überschriften etc. auf englisch