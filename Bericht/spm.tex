\section{Sequential Pattern Mining}\label{spm}

\subsection{Überblick}

\textit{Sequential pattern mining} entdeckt häufige \textit{subsequences} (dt. Teilfolgen) in Datenbanken. Sogenannte \textit{sequence databases} bestehen aus Transaktionen, die jeweils \textit{items} enthalten, welche der Zeit nach geordnet sind. Die Daten lassen sich also mit dem Schema [Transaction/ID, <Ordered Sequence Items>] darstellen.\\
Ein Anwendungsfeld ist die Warenkorbanalyse. Angenommen es wird das Kaufverhalten in einem Supermarkt einen Monat lang beobachtet, dann könnte [Kunde 1, <(Brot, Milch), (Brot, Milch, Tee), (Zucker), (Milch, Salz)>]; [Kunde 2, <(Brot), (Milch, Tee)>] eine Beispiel-Datenbank sein. Kunde 1 war vier mal im beobachteten Monat im Supermarkt einkaufen, wobei Kunde 2 nur zweimal einkaufen war. Der Kunde kann nur eins oder auch mehrere \textit{items} pro Besuch einkaufen. Im Falle von mehreren \textit{items} spricht man von \textit{itemsets}.\\
\textit{Web usage mining} ist das am weitesten verbreitete Anwendungsfeld von \textit{sequential pattern mining} in der Literatur (\cite{lu_ezeife,wang_han,goethals}). Unter der Annahme, dass ein Internetnutzer nur eine Webseite an einem Zeitpunkt aufrufen kann, besteht die Folge von geordneten \textit{items} nur aus einzelnen \textit{items} und nicht aus \textit{itemsets}. Ist also eine Menge von \textit{items I = \{a, b, c, d, e\}} gegeben, die beispielsweise verschiedene Webseiten repräsentieren, so könnte eine Datenbank mit zwei Nutzern folgendermaßen aussehen: [Nutzer 1, <abedcab>]; [Nutzer 2, <edcaa>] (\cite[3:1-3:2]{taxonomy}).

\subsection{Notationen und Definitionen}
Gegeben ist eine Menge von Sequenzen, die eine sequentielle Datenbank $D$ bilden, ein minimum support threshold $min\_sup$ $\xi$ und eine Menge von $k$ eindeutigen items $I=\{i_1,i_2,...,i_k\}$. Das Ziel von sequential pattern mining ist das Finden aller häufig auftretenden Sequenzen $S$ von items aus $I$ in der Datenbank $D$ bei gegebenem $min\_sup$ $\xi$.\\
Im vorliegenden Fall sind die items die verschiedenen Marketing-Kanäle, die hier beispielhaft als $I=\{a,b,c,d,e\}$ dargestellt sind. Ein itemset ist eine nichtleere, ungeordnete Menge von items, zum Beispiel $(eab)$. Lexikographisch geordnete itemsets bilden eine Sequenz, beispielsweise $S=<b(eab)ac(cd)>$.\\
Set Lexicographical Order (\cite{lexico}) kann wie folgt definiert werden. Gegeben ein itemset $t=\{i_1,i_2,...,i_k\}$ mit $k$ unterschiedlichen items und ein weiteres itemset $t'=\{j_1,j_2,...,j_l\}$ mit $l$ unterschiedlichen items mit $i_1\le i_2\le ...\le i_k$ und $j_1\le j_2\le ... \le j_l$, wobei $i_1\le i_2$ bedeutet, dass $i_1$ vor $i_2$ eintritt. Dann gilt $t<t'$, wenn (1) für $h\in\mathbb{N}$, $0\le h\le min\{k,l\}$, $r<h$ und $i_h<j_h$ gilt $i_r=j_r$ oder (2) $k<l$ und $i_1=j_1$, $i_2=j_2,...,i_k=j_k$.\\
Eine Sequenz $\alpha=<\alpha_1\alpha_2...\alpha_m>$ ist Subsequenz einer anderen Sequenz $\beta=<\beta_1\beta_2...\beta_n>$, in Zeichen $\alpha\preceq\beta$, wenn eine injektive, isotone Funktion $f$ existiert, die items in $\alpha$ auf items in $\beta$ abbildet, das heißt (1) $\alpha_i\subseteq f(\alpha_i)$ und (2) wenn $\alpha_i<\alpha_j$ ist, dann ist $f(\alpha_i)<f(\alpha_j)$.


\subsection{Auswahl geeigneter Algorithmen}

In den letzten zwei Jahrzehnten wurden im Forschungsfeld des \textit{sequential pattern mining} eine Vielzahl von Algorithmen entwickelt (\cite{hvsm,lapin,aprioriall,gsp,psp,spam,freespan,prefixspan,wapmine,fsminer,discall,spade,plwap}; \textbf{die die an anderer stelle zitiert werden, hier später löschen}). Im Folgenden sollen die besten Algorithmen für die gegebene Datenlage ausgewählt werden.
