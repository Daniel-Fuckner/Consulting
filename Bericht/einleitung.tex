\section{Einleitung}

Die Interhyp AG ist Vermittler für private Baufinanzierungen. Das heißt, sie wählt aus einem Angebot von verschiedenen Darlehensgebern die optimale Finanzierungsstruktur für einen Kunden aus. Das Unternehmen wurde 1999 basierend auf der Idee, die Baufinanzierungsbranche zu revolutionieren, von den ehemaligen Goldman-Sachs-Bankern Robert Haselsteiner und Marcus Wolsdorf gegründet. Sechs Jahre später eröffnete die Interhyp AG erste Niederlassungen und konnte gleichzeitig den erfolgreichsten deutschen Börsengang des Jahres verzeichnen. Nach weiteren drei Jahren erfolgte die Übernahme durch ING DIRECT, der weltweilt größten und erfolgreichsten Direktbanken-Gruppe. Heute ist die Interhyp AG der größte Vermittler für private Baufinanzierungen in Deutschland, wurde acht mal in Folge als "Bester Baufinananzierer" (Zeitschrift \euro, Ausgabe 08/2013) ausgezeichnet und verfügt über mehr als 60 Beratungsstandorte mit über 1.000 Mitarbeitern.\\
Das primäre Ziel des Marketing der Interhyp AG ist die Kundenakquise. Da etwa 80\% aller Kundenanträge online abgeschickt werden, liegt der Fokus der Marketing-Abteilung auf dem Online-Marketing, das über verschiedene Kanäle verfügt. Beispiele sind die Kooperationen (z.B. mit Immobilienscout24), Suchmaschinen (bezahlte Anzeigen und unbezahlte Ergebnisse), Affiliate Marketing (Netzwerk kleinerer Partnerseiten), Display Advertising (diverse Bannerschaltungen), Newsletter und Social Media (vorrangig Facebook und gutefrage.net). Durch Online-Tracking können die Werbekontakte eines potentiellen Kunden mit der Interhyp AG zusammengefasst werden. So entsteht ein Customer Journey (siehe Abbildung ??), dass zum Abbruch oder im Idealfall zum Ausfüllen eines Onlineantrages führt. An dieser Stelle kommt die Refined Labs GmbH ins Spiel.\\
\textbf{Customer Journey Grafik aus Folien}

%\begin{figure}[H]
	%\centering
  %\includegraphics{Innenkammer.png}
	%\caption{Aufbau der Innenkammer im Ofen aus VDI 2263 Blatt 1:1990-05}
	%\label{customer_journey}
%\end{figure}

Die Refined Labs GmbH ist auf dem Gebiet des Online-Marketing spezialisiert und führender Anbieter für Performance-Marketing-Software. Zum Kundenportfolio zählt unter anderem auch die Interhyp AG, das heißt das Online-Tracking wird von der Refined Labs GmbH durchgeführt und verwaltet. Ein Customer Journey beginnt mit dem ersten Online-Werbekontakt eines potentiellen Kunden mit der Interhyp AG und der damit einhergehenden Erstellung eines Cookies. So können alle weiteren Werbekontakte dem potentiellen Kunden eindeutig zugewiesen werden. Das Tracking endet sobald der potentielle Kunde einen Onlineantrag versendet und damit zum Kunden wird. Wird innerhalb von 90 Tagen kein Onlineantrag versendet, so wird das Cookie automatisch gelöscht und man spricht von einem Abbruch.\\
Die Interhyp AG ist daran interessiert, ob ein Abbruch von Customer Journeys verhindert werden kann. Hier dann die Zielsetzungen einfügen, die wir gelöst haben...........\\
Überblick über alle folgenden Kapitel.\\
\textbf{Quellen: Projektausschreibung und Folien von Frau Gries; wie angeben?}\\
