\documentclass[12pt,a4paper]{scrartcl} 

%Deutsch:
	\usepackage[ngerman]{babel} %Deutsches Datumformat, Umlaute m\"oglich,...
	\usepackage[latin9]{inputenc} %Umlaute direkt eingebbar
	
%Quellenverzeichnis:
	\usepackage[autocite=footnote,uniquename=full,uniquelist=true,backend=bibtex8]{biblatex}
	\usepackage{csquotes} %Hilfspaket für Biblatex
	\bibliography{bib_database.bib} %Datei mit bibliographischen Daten

%Mathematik:
	\usepackage{dsfont} %Symbole
	\usepackage{amsmath} %Umgebung
	\usepackage{amssymb} %Symbole
	\usepackage{bbm} %doppelstreifen bei buchstaben (zb symbol für ganze zahlen \mathbbm{Z})
	
%Grafiken:
	\usepackage{graphics}
	\usepackage{graphicx}
	\usepackage{picinpar} %bilder so einfügen, dass text um bilder weiterläuft
	\usepackage{float} %\begin{figure}[H] => Grafik wird HIER eingefügt!

%Quellcode:
	%\usepackage[numbered,framed]{mcode} %Quellcode darstellen
	
%Englisch:
	%\usepackage[ngerman,english]{babel} %automatisch erzeugte Überschriften etc. auf englisch % Importiere die Einstellungen aus der Präambel

% hier beginnt der eigentliche Inhalt
\begin{document}

% Deckblatt
\begin{titlepage}
	\rmfamily
	\begin{center}
		% Logo
		%\includegraphics[width=0.15\textwidth]{./logo}\\[1cm]
	
		\textsc{\LARGE Statistisches Consulting}\\[1.5cm]

		\textsc{
			\large{	Master Studiengang Statistik\\[0.25cm]
							Institut für Statistik\\[0.25cm]
							Ludwig-Maximilians-Universität München}}\\[0.25cm]
						
		% Title
		\newcommand{\HRule}{\rule{\linewidth}{0.5mm}}
		\HRule \\[0.4cm]
		{\huge \bfseries Statistisches Consulting für die Interhyp AG\\[0.5cm]Marketing im Internet}\\[0.4cm]
		\HRule \\[1.5cm]
		
		% Autoren
		\textbf{Daniel Fuckner} d.fuckner@gmx.de\\
		\textbf{Markus Vogler} markus@vogler-lindau.de\\[1.5cm]
	
		% Betreuer und Projektpartner
		\begin{minipage}{0.4\textwidth}
			\begin{flushleft}
				Projektpartner:\\
				\textbf{Interhyp AG}
			\end{flushleft}
		\end{minipage}
		\hfill
		\begin{minipage}{0.4\textwidth}
			\begin{flushright}
				Betreuer:\\
				\textbf{Dr. Fabian Scheipl}
			\end{flushright}
		\end{minipage}
		
		\vfill

		% Datum
		{\large München, 11.11.2011}
		
	\end{center}
\end{titlepage}

%\thispagestyle{plain}
\pagenumbering{roman} %römische Seitennummerierung

%Abstract
\begin{abstract}
\noindent\subsubsection*{Abstract}
Die Interhyp AG ist Vermittler für private Baufinanzierungen. Das primäre Ziel des Marketing der Interhyp AG ist die Kundenakquise. Da etwa 80\% aller Kundenanträge online abgeschickt werden, liegt der Fokus auf dem Online-Marketing, das über verschiedene Kampagnen verfügt. Beispiele sind Kooperationen mit anderen Unternehmen, bezahlte Anzeigen bei Suchmaschinen oder Bannerschaltungen.\\
Die Refined Labs GmbH ist verantwortlich für das Online-Tracking der Werbekampagnen der Interhyp AG. Durch Online-Tracking werden die Werbekontakte eines potentiellen Kunden zu einem sogenannten Funnel zusammengefasst. Am Ende eines jeden Funnels steht das Ausfüllen eines Onlineantrages oder der Abbruch. Man spricht von konvertierten beziehungsweise nicht-konvertierten Funnels.\\
In dieser Arbeit werden die Daten zunächst anhand deskriptiver Analysen vorgestellt. Außerdem werden Methodik und Ergebnisse eines Sequential Pattern Mining-Algorithmus sowie eines zeitdiskreten Survival-Modells, welches mittels Stochastic Gradient Boosting geschätzt wurde, beschrieben. Zudem wurden die konvertierten und nicht-konvertierten Funnels in Form eines Netzwerkes visualisiert. Die Daten für dieses Netzwerk und ein Programm, dass die interaktive Betrachtung ermöglicht, ist im elektronischen Anhang enthalten.
\end{abstract}

%Inhaltsverzeichnis
\newpage
\tableofcontents

\newpage
\pagenumbering{arabic} %ab hier wieder normale Seitennummerierung

% Hier beginnt der eigentliche Text
\section{Einleitung}

Die Interhyp AG ist Vermittler für private Baufinanzierungen. Das heißt, sie wählt aus einem Angebot von verschiedenen Darlehensgebern die optimale Finanzierungsstruktur für einen Kunden aus. Das Unternehmen wurde 1999 basierend auf der Idee, die Baufinanzierungsbranche zu revolutionieren, von den ehemaligen Goldman-Sachs-Bankern Robert Haselsteiner und Marcus Wolsdorf gegründet. Sechs Jahre später eröffnete die Interhyp AG erste Niederlassungen und konnte gleichzeitig den erfolgreichsten deutschen Börsengang des Jahres verzeichnen. Nach weiteren drei Jahren erfolgte die Übernahme durch ING DIRECT, der weltweilt größten und erfolgreichsten Direktbanken-Gruppe. Heute ist die Interhyp AG der größte Vermittler für private Baufinanzierungen in Deutschland, wurde acht mal in Folge als "Bester Baufinananzierer" (Zeitschrift \euro, Ausgabe 08/2013) ausgezeichnet und verfügt über mehr als 60 Beratungsstandorte mit über 1.000 Mitarbeitern.\\
Das primäre Ziel des Marketing der Interhyp AG ist die Kundenakquise. Da etwa 80\% aller Kundenanträge online abgeschickt werden, liegt der Fokus der Marketing-Abteilung auf dem Online-Marketing, das über verschiedene Kanäle verfügt. Beispiele sind die Kooperationen (z.B. mit Immobilienscout24), Suchmaschinen (bezahlte Anzeigen und unbezahlte Ergebnisse), Affiliate Marketing (Netzwerk kleinerer Partnerseiten), Display Advertising (diverse Bannerschaltungen), Newsletter und Social Media (vorrangig Facebook und gutefrage.net). Durch Online-Tracking können die Werbekontakte eines potentiellen Kunden mit der Interhyp AG zusammengefasst werden. So entsteht ein Customer Journey (siehe Abbildung \ref{customer_journey}), dass zum Abbruch oder im Idealfall zum Ausfüllen eines Onlineantrages führt. An dieser Stelle kommt die Refined Labs GmbH ins Spiel.\\
\textbf{Customer Journey Grafik aus Folien}

%\begin{figure}[H]
	%\centering
  %\includegraphics{Innenkammer.png}
	%\caption{Aufbau der Innenkammer im Ofen aus VDI 2263 Blatt 1:1990-05}
	%\label{customer_journey}
%\end{figure}

Die Refined Labs GmbH ist auf dem Gebiet des Online-Marketing spezialisiert und führender Anbieter für Performance-Marketing-Software. Zum Kundenportfolio zählt unter anderem auch die Interhyp AG, das heißt das Online-Tracking wird von der Refined Labs GmbH durchgeführt und verwaltet. Ein Customer Journey beginnt mit dem ersten Online-Werbekontakt eines potentiellen Kunden mit der Interhyp AG und der damit einhergehenden Erstellung eines Cookies. So können alle weiteren Werbekontakte dem potentiellen Kunden eindeutig zugewiesen werden. Das Tracking endet sobald der potentielle Kunde einen Onlineantrag versendet und damit zum Kunden wird. Wird innerhalb von 90 Tagen kein Onlineantrag versendet, so wird das Cookie automatisch gelöscht und man spricht von einem Abbruch.\\
Die Interhyp AG ist daran interessiert, ob ein Abbruch von Customer Journeys verhindert werden kann. Hier dann die Zielsetzungen einfügen, die wir gelöst haben...........\\
Überblick über alle folgenden Kapitel.\\
\textbf{Quellen: Projektausschreibung und Folien von Frau Gries; wie angeben?}\\
 % Importiere die Einleitung
\section{Datenlage}\label{datenlage}
%\begin{table}[H]
 %   \begin{center}
  %      \begin{tabular}{|c|p{10cm}|}
   %         \hline $ touchpoint $  & Kontaktpunkt einer Beobachtung \\
    %        \hline $ funnel $ & zeitliche Abfolge der Touchpoints einer Beobachtung\\ 
     %       \hline $ campaign $ & Kategorie der Werbeform auf der obersten Ebene, gegebenfalls auch Ebenen darunter \\ 
      %      \hline $ position \in \{1,2,\dots\}$  & Nummer des Touchpoints eines Funnels\\
       %     \hline $ transaction \in \{0,1\}$  & $1$ steht für konvertiert und $0$ für nicht-konvertiert\\
        %    \hline $ funnelLength \in \{1,2,\dots\} $  & Anzahl der Touchpoints eines Funnels\\
         %   \hline
        %\end{tabular} 
    %\end{center}
    %\caption{Definition der wichtigsten Begriffe}
%\end{table}
Die Daten wurden von der Refined Labs GmbH als SQL-Dump bereitgestellt, der eine Größe von circa $13$ Gigabyte hat. Die MySQL-Datenbank enthält die vier Tabellen \textit{project\_out}, \textit{redirects\_short}, \textit{searchFunnel} und \textit{stage2\_transactionHandling}. Mit Hilfe der vorhanden Informationen in \textit{searchFunnel} und \textit{stage2\_transactionHandling} konnten die Kontaktpunkte in \textit{redirects\_short} in konvertierte und nicht-konvertierte Funnels unterteilt werden. In \textit{projects\_out} sind die Kampagnen in Form einer Baumstruktur organisiert. In Absprache mit der Interhyp AG wurden $17$ Kategorien ausgewählt, die sich auf den ersten drei Ebenen dieser Baumstruktur befinden. Anhand von IDs wurde jedem Kontaktpunkt eine dieser Kategorien zugewiesen. Die $17$ Kampagnen und weitere Features, die aus den Daten erzeugt wurden, werden in Kapitel \ref{descriptiv} näher erläutert.\\
Ein Kontaktpunkt ist entweder ein \textit{Click} oder ein \textit{View}. Man spricht von einem \textit{Click}, wenn der potentielle Kunde tatsächlich etwas angeklickt hat, wobei die genaue Definition von der Kampagne abhängt. Ein \textit{View} wird getrackt, wenn ein Banner oder ähnliches lediglich gesehen, aber nicht angeklickt wird. An dieser Stelle wirft die Datenerhebung allerdings ein Problem für die statistischen Analysen auf. Die \textit{Views} werden für alle konvertierten Funnels gespeichert, für die nicht-konvertierten Funnels allerdings nur, wenn diese bei einem anderen Kunden der Refined Labs GmbH konvertieren. Dass heißt, es ist eine systematische Veränderung der Daten gegeben. Deshalb besteht keine Möglichkeit die \textit{Views} in statistische Analysen, die konvertierte und nicht-konvertierte Funnels vergleichen, einzubeziehen. Die \textit{Views} werden lediglich in Kapitel \ref{descriptiv} in einigen Plots betrachtet, die nur konvertierte Funnels enthalten, und von den weiteren Analysen ausgeschlossen.\\
Nach der Vorverarbeitung der Daten liegen ??? konvertierte und ??? nicht-konvertierte Funnels vor, die nur \textit{Clicks} enthalten. Eine nähere Beschreibung der erstellten Features erfolgt in Kapitel \ref{descriptiv}.




\section{Deskriptive Analyse}

\begin{frame}\frametitle{Inhalt}
	\tableofcontents[currentsection,hideallsubsections]
\end{frame}

\begin{frame}\frametitle{Beispiel für einen Auszug aus der Datenbank}
	\begin{table}[H]
		\begin{center}
			\begin{tabular}{|c|l|c|c|c|c|}
				\hline
				ID & Campaign 									 & Transaction & Position & ... \\ \hline\hline
				1  & Affiliate - Partnerprogramm & 0					 & 1		    & ... \\ \hline
				1  & SEM - Brand                 & 0					 & 2		    & ... \\ \hline
				1  & Direct                      & 0					 & 3		    & ... \\ \hline
				1  & Direct                      & 1					 & 4		    & ... \\ \hline
				2  & Display                     & 0					 & 1		    & ... \\ \hline
				2  & SEM - Generisch             & 0					 & 2		    & ... \\ \hline
				2  & Social Media                & 0					 & 3		    & ... \\ \hline
			\end{tabular} 
		\end{center}
	\end{table}
\end{frame}

\begin{frame}\frametitle{Datenlage} 
	\begin{itemize}
		\item SQL-Dump mit Größe von circa $13$ Gigabyte
		\item Einteilung in konvertierte und nicht-konvertierte Funnels
		\item Kampagnen in Form einer Baumstruktur organisiert
		\item Festlegung auf $17$ Kategorien
		\item \textit{Views} liegen in den nicht-konvertierten Funnels nur vor, wenn diese bei einem anderen Kunden der Refined Labs GmbH konvertiert sind
		\item $ 297,963 $ \textit{Clicks} für die konvertierten und $ 9,550,802 $ \textit{Clicks} für die nicht-konvertierten Funnels
		\item Erstellung von Features
	\end{itemize}
\end{frame}

\subsection{Views in den konvertierten Funnels}

\begin{frame}\frametitle{clickCount}
	    \centering\includegraphics[scale=0.39]{clickCountSucc.pdf}
\end{frame}

\begin{frame}\frametitle{hasClicked}
	    \centering\includegraphics[scale=0.39]{hasClickedSucc.pdf}
\end{frame}

\begin{frame}\frametitle{Beschreibung der Kampagnen}
	\begin{table}[H]
		\tiny
		\begin{center}
			\begin{tabular}{|l|p{7cm}|}
				\hline \textbf{Kampagne} & \textbf{Beschreibung}\\ \hline
				\hline Affiliate - Partnerprogramm & Partner, die von der Interhyp AG bereitgestellte Werbemittel wie Rechner, Logo oder Banner einbinden\\
				\hline Affiliate - Rest & Partner, die einen Zinsvergleich bereitstellen, welcher das Zinsangebot der Interhyp AG mit deren Wettbewerbern im Vergleich darstellt\\ 
				\hline Direct & Potentieller Kunde gibt im Browser direkt \textit{www.interhyp.de} ein\\ 
				\hline Display & Bannerschaltungen\\
				\hline E-Mailing & Mails an Interessenten, die schon einen Antrag gestellt oder ein Infopaket angefordert hatten\\
				\hline Generic & Potentieller Kunde kommt über unbezahlten Link zur Interhyp AG\\
				\hline Kooperationen - Focus & \multirow{5}{7cm}{Individuelle Zusammenarbeiten mit größeren Partnern, die je nach Vertrag verschiedene Werbemittel auf ihrer Seite einbinden}\\
				Kooperationen - Immonet & \\
				Kooperationen - Immoscout24 & \\
				Kooperationen - Immowelt & \\
				Kooperationen - Rest & \\
				\hline Newsletter & Regelmäßige Rundschreiben\\
				\hline SEM - Brand & Bezahlte Suchergebnisse, wobei nach \textit{Interhyp} oder ähnlichem gesucht wurde\\
				\hline SEM - Remarketing & Bezahlte Suchergebnisse, wobei der potentielle Kunde bereits zuvor auf der Seite der Interhyp AG war\\
				\hline SEM - Generisch & Bezahlte Suchergebnisse, wobei nach \textit{Baufinanzierung} oder ähnlichem gesucht wurde\\
				\hline SEO & Unbezahlte Suchergebnisse\\
				\hline Social Media & Werbung, vor allem auf \textit{facebook} und \textit{gutefrage.net}\\
				\hline
			\end{tabular} 
		\end{center}
	\end{table}
\end{frame}

\begin{frame}\frametitle{campaign}
	    \centering\includegraphics[scale=0.3]{campaignSucc.pdf}
\end{frame}

\subsection{Vergleich von konvertierten und nicht-konvertierten Funnels}

\begin{frame}\frametitle{weekday}
	    \centering\includegraphics[scale=0.3]{weekday.pdf}
\end{frame}

\begin{frame}\frametitle{hour}
	    \centering\includegraphics[scale=0.3]{hour.pdf}
\end{frame}

\begin{frame}\frametitle{campaign}
	    \centering\includegraphics[scale=0.3]{campaign.pdf}
\end{frame}

\begin{frame}\frametitle{funnelLength}
	    \centering\includegraphics[scale=0.3]{funnelLength_First.pdf}
\end{frame}

\begin{frame}\frametitle{timeSinceFirst}
	    \centering\includegraphics[scale=0.3]{timeSinceFirst_Last.pdf}
\end{frame}

\begin{frame}\frametitle{timeSinceLast}
	    \centering\includegraphics[scale=0.3]{timeSinceLast.pdf}
\end{frame}

\begin{frame}\frametitle{freq}
	    \centering\includegraphics[scale=0.3]{freq.pdf}
\end{frame}

\section{Sequential Pattern Mining}\label{spm}

\subsection{Überblick}

\textit{Sequential pattern mining} entdeckt häufige \textit{subsequences} (dt. Teilfolgen) in Datenbanken. Sogenannte \textit{sequence databases} bestehen aus Transaktionen, die jeweils \textit{items} enthalten, welche der Zeit nach geordnet sind. Die Daten lassen sich also mit dem Schema [Transaction/ID, <Ordered Sequence Items>] darstellen.\\
Ein Anwendungsfeld ist die Warenkorbanalyse. Angenommen es wird das Kaufverhalten in einem Supermarkt einen Monat lang beobachtet, dann könnte [Kunde 1, <(Brot, Milch), (Brot, Milch, Tee), (Zucker), (Milch, Salz)>]; [Kunde 2, <(Brot), (Milch, Tee)>] eine Beispiel-Datenbank sein. Kunde 1 war vier mal im beobachteten Monat im Supermarkt einkaufen, wobei Kunde 2 nur zweimal einkaufen war. Der Kunde kann nur eins oder auch mehrere \textit{items} pro Besuch einkaufen. Im Falle von mehreren \textit{items} spricht man von \textit{itemsets}.\\
\textit{Web usage mining} ist das am weitesten verbreitete Anwendungsfeld von \textit{sequential pattern mining} in der Literatur (\cite{lu_ezeife,wang_han,goethals}). Unter der Annahme, dass ein Internetnutzer nur eine Webseite an einem Zeitpunkt aufrufen kann, besteht die Folge von geordneten \textit{items} nur aus einzelnen \textit{items} und nicht aus \textit{itemsets}. Ist also eine Menge von \textit{items I = \{a, b, c, d, e\}} gegeben, die beispielsweise verschiedene Webseiten repräsentieren, so könnte eine Datenbank mit zwei Nutzern folgendermaßen aussehen: [Nutzer 1, <abedcab>]; [Nutzer 2, <edcaa>] (\cite[3:1-3:2]{taxonomy}).

\subsection{Auswahl geeigneter Algorithmen}

In den letzten zwei Jahrzehnten wurden im Forschungsfeld des \textit{sequential pattern mining} eine Vielzahl von Algorithmen entwickelt (\cite{hvsm,lapin,aprioriall,gsp,psp,spam,freespan,prefixspan,wapmine,fsminer,discall,spade,plwap}; \textbf{die die an anderer stelle zitiert werden, hier später löschen}). Im Folgenden sollen die besten Algorithmen für die gegebene Datenlage ausgewählt werden.

\section{Zeitdiskretes Proportional-Hazards-Modell von Cox}

Aufgrund der  in Kapitel \ref{datenlage} beschriebenen Datenlage erscheint die Anwendung eines Modells aus dem Feld der Lebensdaueranalyse intuitiv. Für solch ein Regressionsmodell müssen die Daten in der Form $(t_i, \delta_i, x_i(t))$, $i = 1,..,n$ vorliegen.  
\section{Visualisierung der Daten anhand eines Netzwerkes}\label{network}

Während Sequential Pattern Mining lediglich häufige Sequenzen in den Daten findet, sollen die Daten zusätzlich noch in Form eines Netzwerkes dargestellt werden. Dies ermöglicht die Visualisierung des kompletten Datensatzes und liefert weitere Informationen bezüglich von Mustern in den Daten.\\
Zum bessern Verständnis sollen an dieser Stelle einige elementare Grundlagen der Graphentheorie vorgestellt werden. Ein geordneter Graph $G=(V,E)$ besteht aus einer Menge $V$ von Knoten und einer Menge $E$ von Kanten. Eine Kante $e_i \in E$ besteht aus einem geordneten Paar von zwei Knoten $(v_j,v_k)$, wobei $v_j,v_k \in V$. Das heißt eine Kante stellt die geordnete Verbindung zwischen zwei Knoten dar \cite[16]{network_data}.\\
Ausgehen von einem Startpunkt führen Kanten zu den $17$ Kampagnen der ersten Position. Diese $17$ Kampagnen entsprechen jeweils einem Knoten. Die Namen der Knoten setzen sich aus dem Namen der Kampagne und $\_1$ für Position $1$ zusammen. Von jeder Kampagne führen dann Kanten zu $Succ\_1$, $Fail\_1$ und zu den $17$ Kampagnen der zweiten Position. Die Knoten der zweiten Position setzen sich wiederum aus dem Namen der Kampagne und $\_2$ zusammen. Von dort führen wieder Kanten zu $Succ\_2$, $Fail\_2$ und den Kampagnen der dritten Position. Dieses Prinzip setzt sich für die weiteren Positionen fort.\\
Jeder Knoten hat eine absolute Anzahl von potentiellen Kunden. Anhand imaginärer Zahlen soll hier das Prinzip kurz dargestellt werden. Wenn $50000$ Nutzer als ersten Kontakt $Direct$ haben, so hat $Direct\_1$ ein Gewicht von $50000$. Diese verteilen sich nun auf die Knoten, die mit $Direct\_1$ durch eine Kante verbunden sind. Wenn die Kante zwischen $Direct\_1$ und $Direct\_2$ beispielsweise ein Gewicht von $10000$ hat, so haben $10000$ Nutzer als ersten und zweiten Kontakt $Direct$. Wenn die Kante zwischen $Direct\_1$ und $Succ\_1$ ein Gewicht von $500$ hat, so bedeutet das, dass $500$ Nutzer als ersten Kontakt $Direct$ haben und danach direkt konvertieren.\\
Um aussagekräftigere Ergebnisse zu erzielen, wurden für die Gewichtung der Kanten nicht die absoluten Anzahlen gewählt. Stattdessen wurden zwei Strategien verfolgt. Der erste Ansatz war, alle Kanten, die einen Knoten verlassen, bezüglich der relativen Häufigkeit zu gewichten, so dass sie in der Summe $1$ ergeben. Wenn nun das Gewicht der Kante zwischen $Direct\_2$ und $Fail\_2$ $0.2$ entspricht, so sind $20 \%$ der Nutzer, die als zweiten Kontakt $Direct$ haben, nach dem zweiten Kontaktpunkt konvertiert. Diese Gewichtung erlaubt somit eine relative Betrachtung der Kanten die eine Kampagne verlassen.\\
In einem zweiten Ansatz wurden die Kanten dahingehend gewichtet, dass alle Kanten die in einen Knoten gehen bezüglich der relativen Häufigkeit gewichtet werden. Betrachtet man beispielsweise $Succ\_2$, so kann man erkennen, aus welchen Kampagnen der zweiten Position sich die konvertierten Funnels der Länge $2$ zusammen setzen. Hat die Kante zwischen $Direct\_2$ und $Succ\_2$ nun einen Gewicht von $0.2$, so haben $20 \%$ der konvertierten Funnels der Länge $2$ als letzten Kontaktpunkt vor der Konvertiertung $Direct$.\\
Das Netzwerk mit den gewichteten Kanten wird in \textit{R} mit dem Paket \textit{rgexf} \cite{rgexf} erzeugt und als \textit{gexf}-Datei exportiert. Daraufhin wird die Datei in das Open-Source Programm \textit{Gephi} \cite{gephi_bastian} geladen. Dort werden die Daten dann als Graph dargestellt. Abbildung \ref{graphbegin} zeigt die räumliche Anordnung der Knoten und Kanten nach dem ersten Einlesen der \textit{gexf}-Datei in \textit{Gephi}, die noch willkürlich ist. Die farbigen Punkte sind die Knoten und die Linien die Kanten.
\begin{figure}[H]
	\centering\includegraphics[scale=0.6]{graphbegin.pdf}\caption{Anordnung der Knoten und Kanten nach dem Einlesen}\label{graphbegin}
\end{figure}
Für die Berechnung der räumlichen Anordnung der Knoten und Kanten stehen in \textit{Gephi} einige Algorithmen zur Verfügung. Diese berechnen die Anordung der Knoten anhand der Anziehungs- beziehungsweise Abstoßungskraft der Knoten, die aus den relativen Häufigkeiten resultieren. Stärker verbundene Knoten liegen damit näher beieinander als schwach verbundene. Das heißt Knoten, die überhaupt keine Verbindung enthalten, liegen weiter auseinander. Ein Beispiel wäre $Direct\_1$ und $Direct\_3$, da man von der ersten Position nicht direkt zur dritten Position springen kann, sondern zunächst einen Kontaktpunkt an der zweiten Position benötigt. Durch die Anwendung eines Algorithmus ergibt sich somit eine lineare Struktur, die die Positionen aneinander reiht.\\
Für diese Arbeit wurden der Algorithmus \textit{Force Atlas 2} \cite{forceatlas2} und der Algorithmus nach \textit{Yifan Hu} \cite{yifanhu} verwendet, wobei \textit{Yifan Hu} das Netzwerk besser in einzelne Ebenen einteilen kann. Das heißt die Positionen sind hier räumlich deutlicher getrennt. Für die Präsentation der Ergebnisse in Kapitel \ref{resultsnetwork} wurde die räumliche Anordnung der Knoten allerdings noch manuell bearbeitet, um die Ergebnisse anschaulicher zu machen.\\
Das Netzwerk kann in \textit{Gephi} interaktiv bearbeitet werden. Die Daten sowie das Programm sind im elektronischen Anhang enthalten und werden in Kapitel \ref{anhang} näher beschrieben.




\input{ergebnisse.tex}
\section{Zusammenfassung}

\begin{frame}\frametitle{Inhalt}
	\tableofcontents[currentsection,hideallsubsections]
\end{frame}

\section{Elektronischer Anhang}\label{anhang}

\begin{figure}[H]
\dirtree{% dieses Kommentar-Zeichen ist noetig, da der erste Charakter in der Umgebung ein '.' sein muss
.1 consulting/.
.2 r\_results/.
}
\caption{Verzeichnisstruktur des Projekts}\label{verz}
\end{figure}

\newpage
\listoftables
\listoffigures
\newpage
\printbibliography

\end{document}
