\section{Zusammenfassung}\label{zusammenfassung}

An dieser Stelle sollen die angewendeten Methoden und Ergebnisse noch einmal kurz zusammengefasst werden.\\
Es wurde ein zeitdiskretes Survival-Modell aufgestellt, dass für jede Beobachtung an jeder Position die Konvertierungswahrscheinlichkeit schätzt. Dieses Modell kann als Klassifikator in konvertierte und nicht-konvertierte Funnels fungieren. Die Prognosegüte konnte vor allem durch die Anwendung von Stochastic Gradient Boosting und die Einführung des Offsets verbessert werden. Außerdem liefern die Marginalen Effekte der Features eine gute Interpretationsmöglichkeit der Prädiktorfunktion.\\
Die wichtigsten Ergebnisse an dieser Stelle sind, dass die Kampagnen \textit{Affiliate - Rest}, \textit{E-Mailing}, \textit{SEM - Brand}, \textit{Direct} und \textit{Generic} im Vergleich zu den restlichen Kampagnen besonders gut für die Konvertierung funktionieren. Außerdem haben Funnels mit einer längeren Beobachtungsspanne und einer niedrigeren Frequenz von Kontaktpunkten eine höhere Konvertierungswahrscheinlichkeit.\\
Die Einführung von Interaktionen in das Modell hat zu einer Verschlechterung der Prognoseleistung geführt und wurde deshalb nicht weiter verfolgt. Außerdem konnten die \textit{Views} aufgrund der Datenlage in den statistischen Analysen nicht berücksichtigt werden.\\
Mit dem Sequential Pattern Mining-Algorithmus sollten häufige Sequenzen in den Funnels entdeckt werden. Von den resultierenden Ergebnissen sind die \textit{Direct}-Sequenzen in den konvertierten und die \textit{Affiliate - Partnerprogramm}-Sequenzen in den nicht-konvertierten Funnels erwähnenswert. Allerdings ist zu berücksichtigen, dass zwischen den einzelnen Elementen der Sequenzen auch andere Kampagnen erlaubt sind. Ein weiterer Ansatz wäre nur diejenigen Funnels zu berücksichtigen, die die Sequenzen in der exakten Reihenfolge ohne Lücken vorweisen. Bei der gegebenen Datenlage würde das allerdings zu einer drastischen Reduzierung des Supports der Sequenzen führen, zumal der Support der häufigen Sequenzen aufgrund der überwiegenden Anzahl an kurzen Funnels ohnehin nicht sehr groß ist. Um interessantere Ergebnisse zu erlangen, wurden die Daten deshalb noch in Form eines Netzwerkes visualisiert.\\
Dieses Netzwerk erlaubt die Visualisierung der gesamten Daten. Außerdem können durch die Gewichtung der Kanten mit den relativen Ein- und Ausgängen wertvolle Informationen aus den Daten gezogen werden. Mit Hilfe des, in diesem Bericht bereitgestellten, Tutorials (siehe Kapitel \ref{anhang}) ist ein interaktives Arbeiten mit dem Netzwerk sowie eine weiterführende Erforschung der Daten möglich.